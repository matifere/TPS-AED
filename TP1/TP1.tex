\documentclass[10pt,a4paper]{article}
\usepackage[spanish,activeacute,es-tabla]{babel}
\usepackage[utf8]{inputenc}
\usepackage{ifthen}
\usepackage{listings}
\usepackage{dsfont}
\usepackage{subcaption}
\usepackage{amsmath}
\usepackage[strict]{changepage}
\usepackage[top=1cm,bottom=2cm,left=1cm,right=1cm]{geometry}%
\usepackage{color}%
\usepackage{amsmath}
\input{AEDmacros}
\usepackage{caratula} % Version modificada para usar las macros de algo1 de ~> https://github.com/bcardiff/dc-tex


\titulo{Trabajo práctico 1}
\subtitulo{Especificación y WP}

\fecha{\today}

\materia{Algoritmos y Estructuras de Datos - DC - UBA}
\grupo{Grupo AJMS}

\integrante{Ferechian, Matías}{693/23}{matifere@gmail.com}
\integrante{Nestmann, Sofía}{366/23}{sofianestmann@gmail.com}
\integrante{Mirasson, Javier}{594/23}{javierestebanmn@gmail.com}
\integrante{Ramirez, Ana}{931/23}{correodeanar@gmail.com}
% Pongan cuantos integrantes quieran

% Declaramos donde van a estar las figuras
% No es obligatorio, pero suele ser comodo
\graphicspath{{../static/}}

\begin{document}

\maketitle

\section{Especificación}

%
% EJERCICIO 1
%

\subsection{grandesCiudades}

\begin{proc}{grandesCiudades}{\In ciudades : \TLista{Ciudad}}{\TLista{Ciudad}\{ } %la macros no anda muy bien y no genera los corchetes
    \requiere{\{\True\}}
    \asegura{\{ $ \paraTodo{i}{\ent}{ %no estoy muy seguro de si aca va un para todo o un existe
    		\left( 0 \leq i \textless |ciudades| \right) \wedge
    		 \left( \left( ciudades[i] \in res \right) \implicaLuego \left( ciudades[i]_{1} \textgreater 50000 \right)  \right)
    		} $ \}}
\end{proc} \} %aca lo mismo que antes

\subsection{sumaDeHabitantes}

\begin{proc}{sumaDeHabitantes}{\In menoresDeCiudades : \TLista{Ciudad}, \In mayoresDeCiudades : \TLista{Ciudad}}{\TLista{Ciudad}\{ } 
	\requiere{\{ 
		$\left( |menoresDeCiudades| = |mayoresDeCiudades| \right) \yLuego
		 \left( \paraTodo{i}{\ent}{\left( 0 \leq i \textless |menoresDeCiudades| \right) \wedge \left( menoresDeCiudades[i]_{0} = mayoresDeCiudades[i]_{0} \right)} \right)
		$ \}}
	\asegura{\{ 
		
		 \}}
\end{proc} \} 




\end{document}

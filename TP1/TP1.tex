\documentclass[10pt,a4paper]{article}
\usepackage[spanish,activeacute,es-tabla]{babel}
\usepackage[utf8]{inputenc}
\usepackage{ifthen}
\usepackage{listings}
\usepackage{dsfont}
\usepackage{subcaption}
\usepackage{amsmath}
\usepackage[strict]{changepage}
\usepackage[top=1cm,bottom=2cm,left=1cm,right=1cm]{geometry}%
\usepackage{color}%
\usepackage{amsmath}
\input{AEDmacros}
\usepackage{caratula} % Version modificada para usar las macros de algo1 de ~> https://github.com/bcardiff/dc-tex


\titulo{Trabajo práctico 1}
\subtitulo{Especificación y WP}

\fecha{\today}

\materia{Algoritmos y Estructuras de Datos - DC - UBA}
\grupo{Grupo AJMS}

\integrante{Ferechian, Matías}{693/23}{matifere@gmail.com}
\integrante{Nestmann, Sofía}{366/23}{sofianestmann@gmail.com}
\integrante{Mirasson, Javier}{594/23}{javierestebanmn@gmail.com}
\integrante{Ramirez, Ana}{931/23}{correodeanar@gmail.com}
% Pongan cuantos integrantes quieran

% Declaramos donde van a estar las figuras
% No es obligatorio, pero suele ser comodo
\graphicspath{{../static/}}

\begin{document}

\maketitle

\section{Especificación}

%
% EJERCICIO 1
%

\subsection{grandesCiudades}

\begin{proc}{grandesCiudades}{\In ciudades : \TLista{Ciudad}}{\TLista{Ciudad}\{ } %la macros no anda muy bien y no genera los corchetes
    \requiere{\{\True\}}
    \asegura{\{ $ \paraTodo{i}{\ent}{ %no estoy muy seguro de si aca va un para todo o un existe
    		\left( 0 \leq i \textless |ciudades| \right) \wedge
    		 \left( \left( ciudades[i] \in \res \right) \implicaLuego \left( ciudades[i]_{1} \textgreater 50000 \right)  \right)
    		} $ \}}
\end{proc} \} %aca lo mismo que antes

\subsection{sumaDeHabitantes}

\begin{proc}{sumaDeHabitantes}{\In menoresDeCiudades : \TLista{Ciudad}, \In mayoresDeCiudades : \TLista{Ciudad}}{\TLista{Ciudad}\{ } 
	\requiere{\{ 
		$\left( |menoresDeCiudades| = |mayoresDeCiudades| \right) \yLuego \left( \paraTodo{i,j}{\ent_{\geq 0}}{ 0 \leq i,j < |menoresDeCiudades| \wedge menoresDeCiudades[i]_{0} = mayoresDeCiudades[j]_{0} }  \right) $
		\}} 
	\asegura{\{ 
		\paraTodo{m,n}{\ent}{ $\left 
			( 0 \leq m,n < |menoresDeCiudades| 
			\right)$  $\wedge$
			 $\left(
			  menoresDeCiudades[n]_{0} = mayoresDeCiudades[m]_{0} 
			  \right) 
			  \wedge \\ 
			  \left( \langle menoresDeCiudades[n]_{0}, menoresDeCiudades[n]_{1} + mayoresDeCiudades[m]_{1} \rangle \in \res 
			  \right)$}
		 \}}
\end{proc} \}

\subsection{hayCamino}

\begin{proc}{hayCamino}{\In distancias : \TLista{\TLista{\ent}}, \In desde : \ent, \In hasta : \ent}{\bool \{ } 
	\requiere{\{ (\paraTodo{i,j}{\ent}{ $\left( 0 \leq i,j,desde,hasta < \sqrt{ |distancias| } \right)$ $\yLuego$ $\left( \left( i = j\right) \longrightarrow \left( distancias[i][j] = 0 \right) \right)$ $\wedge$ $\left( distancias[i][j] = distancias[j][i] \right)$} \}}
	
	%HAY QUE PREGUNTAR EL TEMA DEL LARGO DE LA MATRIZ
	
	    \asegura{
		\{ 
		\res = \True \leftrightarrow 
		\existe{p}{\TLista{\ent}}{
			$\left( p[0]=desde \right)$
			$\wedge$
			$\left( p[|p|-1]=hasta \right)$
			$\wedge$ \paraTodo{k}{\ent}{ \ \ \ \ \ \  $\left( 0 \leq k < |p|-1 \right)$  $\rightarrow$ distancias[p[k]][p[k+1]]\textgreater 0}}
		\}
	}
	\}
\end{proc}

\subsection{cantidadCaminosNSaltos}

\begin{proc}{cantidadCaminosNSaltos}{\Inout conexión : \TLista{\TLista{\ent}}, \In n : \ent}{ \{ } 
	\requiere{\{ \True\}}
	\asegura{\{\True\}}
	\}
\end{proc}

\subsection{caminoMinimo}

\begin{proc}{hayCamino}{\In origen : \ent, \In destino : \ent, \In distancias : \TLista{\TLista{\ent}}}{\TLista{\ent} \{ } 
	\requiere{\{ \True \}}
	\asegura{\{\True\}}
	\}
\end{proc}

\section{Demostraciones de correctitud}



\end{document}

\documentclass[10pt,a4paper]{article}
\usepackage[spanish,activeacute,es-tabla]{babel}
\usepackage[utf8]{inputenc}
\usepackage{ifthen}
\usepackage{listings}
\usepackage{dsfont}
\usepackage{subcaption}
\usepackage{amsmath}
\usepackage[strict]{changepage}
\usepackage[top=1cm,bottom=2cm,left=1cm,right=1cm]{geometry}%
\usepackage{color}%
\usepackage{amsmath}
\usepackage[spanish,activeacute,es-tabla]{babel}
\usepackage[utf8]{inputenc}
\usepackage{ifthen}
\usepackage{listings}
\usepackage{dsfont}
\usepackage{subcaption}
\usepackage{amsmath}
\usepackage[strict]{changepage}
\usepackage[top=1cm,bottom=2cm,left=1cm,right=1cm]{geometry}%
\usepackage{color}%
\newcommand{\tocarEspacios}{%
	\addtolength{\leftskip}{3em}%
	\setlength{\parindent}{0em}%
}

% Especificacion de procs

\newcommand{\In}{\textsf{in }}
\newcommand{\Out}{\textsf{out }}
\newcommand{\Inout}{\textsf{inout }}

\newcommand{\encabezadoDeProc}[4]{%
	% Ponemos la palabrita problema en tt
	%  \noindent%
	{\normalfont\bfseries\ttfamily proc}%
	% Ponemos el nombre del problema
	\ %
	{\normalfont\ttfamily #2}%
	\
	% Ponemos los parametros
	(#3)%
	\ifthenelse{\equal{#4}{}}{}{%
		% Por ultimo, va el tipo del resultado
		\ : #4}
}

\newenvironment{proc}[4][res]{%
    
    % El parámetro 1 (opcional) es el nombre del resultado
    % El parámetro 2 es el nombre del problema
    % El parámetro 3 son los parámetros
    % El parámetro 4 es el tipo del resultado
    % Preambulo del ambiente problema
    % Definimos los comandos requiere, asegura, modifica y aux
    \newcommand{\requiere}[2][]{%
        {\normalfont\bfseries\ttfamily requiere}%
        \ifthenelse{\equal{##1}{}}{}{\ {\normalfont\ttfamily ##1} :}\ %
        {##2}%
        {\normalfont\bfseries\,\par}%
    }
    \newcommand{\asegura}[2][]{%
        {\normalfont\bfseries\ttfamily asegura}%
        \ifthenelse{\equal{##1}{}}{}{\ {\normalfont\ttfamily ##1} :}\ %
        {##2}%
        {\normalfont\bfseries\,\par}%
    }
    \renewcommand{\aux}[4]{%
        {\normalfont\bfseries\ttfamily aux\ }%
        {\normalfont\ttfamily ##1}%
        \ifthenelse{\equal{##2}{}}{}{\ (##2)}\ : ##3\, = ##4%
        {\normalfont\bfseries\,;\par}%
    }
    \renewcommand{\pred}[3]{%
        {\normalfont\bfseries\ttfamily pred }%
        {\normalfont\ttfamily ##1}%
        \ifthenelse{\equal{##2}{}}{}{\ (##2) }%
        {%
        \begin{adjustwidth}{+5em}{}
            ##3
        \end{adjustwidth}
        }%
        {\normalfont\bfseries\,\par}%
    }
    
    \newcommand{\res}{#1}
    \vspace{1ex}
    \noindent
    \encabezadoDeProc{#1}{#2}{#3}{#4}
    % Abrimos la llave
    {%
    \par%
    \tocarEspacios
}
{%
    % Cerramos la llave
    }%
    \vspace{1ex}
}



\newcommand{\aux}[4]{%
	{\normalfont\bfseries\ttfamily\noindent aux\ }%
	{\normalfont\ttfamily #1}%
	\ifthenelse{\equal{#2}{}}{}{\ (#2)}\ : #3\, = \ensuremath{#4}%
	{\normalfont\bfseries\,;\par}%
}

\newcommand{\pred}[3]{%
	{\normalfont\bfseries\ttfamily\noindent pred }%
	{\normalfont\ttfamily #1}%
	\ifthenelse{\equal{#2}{}}{}{\ (#2) }%
	\{%
	\begin{adjustwidth}{+2em}{}
		\ensuremath{#3}
	\end{adjustwidth}
	\}%
	{\normalfont\bfseries\,\par}%
}

% Tipos

\newcommand{\nat}{\ensuremath{\mathds{N}}}
\newcommand{\ent}{\ensuremath{\mathds{Z}}}
\newcommand{\float}{\ensuremath{\mathds{R}}}
\newcommand{\bool}{\ensuremath{\mathsf{Bool}}}
\newcommand{\cha}{\ensuremath{\mathsf{Char}}}
\newcommand{\str}{\ensuremath{\mathsf{String}}}

% Logica

\newcommand{\True}{\ensuremath{\mathrm{true}}}
\newcommand{\False}{\ensuremath{\mathrm{false}}}
\newcommand{\Then}{\ensuremath{\rightarrow}}
\newcommand{\Iff}{\ensuremath{\leftrightarrow}}
\newcommand{\implica}{\ensuremath{\longrightarrow}}
\newcommand{\IfThenElse}[3]{\ensuremath{\mathsf{if}\ #1\ \mathsf{then}\ #2\ \mathsf{else}\ #3\ \mathsf{fi}}}
\newcommand{\yLuego}{\land _L}
\newcommand{\oLuego}{\lor _L}
\newcommand{\implicaLuego}{\implica _L}

\newcommand{\cuantificador}[5]{%
	\ensuremath{(#2 #3: #4)\ (%
		\ifthenelse{\equal{#1}{unalinea}}{
			#5
		}{
			$ % exiting math mode
			\begin{adjustwidth}{+2em}{}
				$#5$%
			\end{adjustwidth}%
			$ % entering math mode
		}
		)}
}

\newcommand{\existe}[4][]{%
	\cuantificador{#1}{\exists}{#2}{#3}{#4}
}
\newcommand{\paraTodo}[4][]{%
	\cuantificador{#1}{\forall}{#2}{#3}{#4}
}

%listas

\newcommand{\TLista}[1]{\ensuremath{seq \langle #1\rangle}}
\newcommand{\lvacia}{\ensuremath{[\ ]}}
\newcommand{\lv}{\ensuremath{[\ ]}}
\newcommand{\longitud}[1]{\ensuremath{|#1|}}
\newcommand{\cons}[1]{\ensuremath{\mathsf{addFirst}}(#1)}
\newcommand{\indice}[1]{\ensuremath{\mathsf{indice}}(#1)}
\newcommand{\conc}[1]{\ensuremath{\mathsf{concat}}(#1)}
\newcommand{\cab}[1]{\ensuremath{\mathsf{head}}(#1)}
\newcommand{\cola}[1]{\ensuremath{\mathsf{tail}}(#1)}
\newcommand{\sub}[1]{\ensuremath{\mathsf{subseq}}(#1)}
\newcommand{\en}[1]{\ensuremath{\mathsf{en}}(#1)}
\newcommand{\cuenta}[2]{\mathsf{cuenta}\ensuremath{(#1, #2)}}
\newcommand{\suma}[1]{\mathsf{suma}(#1)}
\newcommand{\twodots}{\ensuremath{\mathrm{..}}}
\newcommand{\masmas}{\ensuremath{++}}
\newcommand{\matriz}[1]{\TLista{\TLista{#1}}}
\newcommand{\seqchar}{\TLista{\cha}}

\renewcommand{\lstlistingname}{Código}
\lstset{% general command to set parameter(s)
	language=Java,
	morekeywords={endif, endwhile, skip},
	basewidth={0.47em,0.40em},
	columns=fixed, fontadjust, resetmargins, xrightmargin=5pt, xleftmargin=15pt,
	flexiblecolumns=false, tabsize=4, breaklines, breakatwhitespace=false, extendedchars=true,
	numbers=left, numberstyle=\tiny, stepnumber=1, numbersep=9pt,
	frame=l, framesep=3pt,
	captionpos=b,
}

\usepackage{caratula} % Version modificada para usar las macros de algo1 de ~> https://github.com/bcardiff/dc-tex


\titulo{Trabajo práctico 1}
\subtitulo{Especificación y WP}

\fecha{\today}

\materia{Algoritmos y Estructuras de Datos - DC - UBA}
\grupo{Grupo AJMS}

\integrante{Ferechian, Matías}{693/23}{matifere@gmail.com}
\integrante{Nestmann, Sofía}{366/23}{sofianestmann@gmail.com}
\integrante{Mirasson, Javier}{594/23}{javierestebanmn@gmail.com}
\integrante{Ramirez, Ana}{931/23}{correodeanar@gmail.com}
% Pongan cuantos integrantes quieran

% Declaramos donde van a estar las figuras
% No es obligatorio, pero suele ser comodo
\graphicspath{{../static/}}

\begin{document}

\maketitle

%
%IMPLEMENTAR ESTA EN EL RANGO (para las matrices, hacer un pred)
%

\section{Especificación}

profe si estas viendo esto es porque me olvide de borrarlo
%
% EJERCICIO 1
%

\subsection{grandesCiudades}

\begin{proc}{grandesCiudades} {\In ciudades : \TLista{Ciudad}}{\TLista{Ciudad}\{ }
	%no hay ciudades repetidas

	\requiere{\{ noHayNombresRepetidos(ciudades)\}}

	%se permuto el implica luego

	\asegura{\{ $ \paraTodo{i}{\ent}{
				\left( 0 \leq i < |ciudades| \right) \implicaLuego
				\left( \left( ciudades[i]_{1} > 50000 \right)
				\implicaLuego
				\left( ciudades[i] \in \res \right)  \right)
			} $ \}}
\end{proc} \}

\pred{noHayNombresRepetidos}{lista: \TLista{Ciudad}}{
	\paraTodo{i,j}{\ent}{\left( 0 \leq i,j < |lista| \right) \implicaLuego \left( lista[i]_{0} \neq lista[j]_{0} \right)}
}

\subsection{sumaDeHabitantes}

%diosmio habia hecho cualquiera
%preguntar si el requiere ahora funca (deberia quedar como una funcion biyectiva)

%no hay repetidos y el implica tienen que ser positivos

\begin{proc}{sumaDeHabitantes}{\In menoresDeCiudades : \TLista{Ciudad}, \In mayoresDeCiudades : \TLista{Ciudad}}
	{\TLista{Ciudad}\{ }
	\requiere{ $\{
			\left( |menoresDeCiudades| = |mayoresDeCiudades| \right)
			\wedge
			noHayNombresRepetidos(menoresDeCiudades) \wedge \\ noHayNombresRepetidos(mayoresDeCiudades) \wedge
			( \paraTodo{i}{\ent}{
				\left( 0 \leq i < |menoresDeCiudades| \right) \implicaLuego \existe{!j}{\ent}{ \left( 0 \leq j < |menoresDeCiudades| \right) \yLuego \left( menoresDeCiudades[i]_{0} = mayoresDeCiudades[j]_{0} \right) }
			}  )
			\}$ }

	%aca se reemplazo ciudades por res directamente

	\asegura{$\{\paraTodo{n,m}{\ent}{ \left(
				0 \leq n,m < |menoresDeCiudades|
				\right)  \implicaLuego
				\left(
				menoresDeCiudades[n]_{0} = mayoresDeCiudades[m]_{0}
				\right)
				\implicaLuego \\
				( \left( res[n]_{1} = menoresDeCiudades[n]_{1} + mayoresDeCiudades[m]_{1} \right) \wedge \\ \left( res[n]_{0} = menoresDeCiudades[n]_{0} \right)
				)}\}$}
\end{proc} \} %Acá preguntar si esta bien el para todo o es para todo i existe k 

\subsection{hayCamino}

%la correccion dice que le faltan cosas al requiere pero estan ahi PREGUNTAR
%se cambio el yLuego por implicaLuego
%se agrego que |p|>=1
%falta cuadradas

\begin{proc}{hayCamino}{\In distancias : \TLista{\TLista{\ent}}, \In desde : \ent, \In hasta : \ent}{\bool \{ }
	\requiere{
		$\{
			\left( 0 \leq desde,hasta < |distancias| \right) \wedge matrizCuadradaSimetrica(distancias)
			\}$
	}

	\asegura{
		$\{
			\res = \True \ \leftrightarrow
			\existe{p}{\TLista{\ent}}{
				\left( |p| > 1 \right) \yLuego (
				\left( p[0] = desde \right) \wedge
				\left( p[|p|-1] = hasta \right) \wedge
				\paraTodo{k}{\ent}{
					(0 \leq k < |p|-1) \implicaLuego (distancias[p[k]][p[k+1]] > 0)
				}
				)
			}  \}$}
\end{proc}
\}

\pred{matrizCuadradaSimetrica}{matriz: \TLista{\TLista{\ent}}}{
	\paraTodo{i,j,k}{\ent}{
		\left( 0 \leq i,j,k < |matriz| \right) \yLuego \left( 0 \leq i,j < |matriz[k]| \right)  \implicaLuego \\ \left( \left( i = j \right) \longrightarrow \left( matriz[i][j] = 0 \right) \right) \wedge \left( matriz[i][j] = matriz[j][i] \right)
	}
}

\subsection{cantidadCaminosNSaltos}
\paragraph{Para la siguiente especificación tendremos en cuenta que:}{Dada la matriz de orden 1:}

\[
	M_{1} = \begin{bmatrix} 0 & 1 & 0 \\	1 & 0 & 1 \\	0 & 1 & 0 \end{bmatrix} \]
\text{ \ \ \ \ Si queremos encontrar la matriz $M_{2}$ de orden 2, nos queda que:}\\
\[
	M_{2} = M_{1} \times M_{1} =
	\begin{bmatrix}	0 & 1 & 0 \\	1 & 0 & 1 \\	0 & 1 & 0\end{bmatrix}
	\times
	\begin{bmatrix}	0 & 1 & 0 \\	1 & 0 & 1 \\	0 & 1 & 0\end{bmatrix}
	=
	\begin{bmatrix}	1 & 0 & 1 \\	0 & 2 & 0 \\	1 & 0 & 1\end{bmatrix}
\]

\sloppy{Luego $M_{2}$ contiene la cantidad de 2-saltos asociados a cada par i,j que se encuentre dentro de la matriz}
\\

%el requiere y el asegura son falsos

\begin{proc}{cantidadCaminosNSaltos}{\Inout conexión : \TLista{\TLista{\ent}}, \In n : \ent}{ \{

		%se corrigio un parentesis y se cambio el yluego por un implicaluego

		\requiere{
			$\{ \left( n>0 \ \wedge matrizCuadradaSimetrica(conexi \acute{o} n)\right) \yLuego
				\paraTodo{i,j}{\ent}{
					\left(0 \leq i,j < |conexi \acute{o} n| \right) \implicaLuego \left(0 \leq conexi \acute{o} n[i][j] \leq 1\right)
				} \wedge \left( C_{0} = conexi \acute{o} n \right)
				\}$
		}

		%hay que ponerle que p[0] = matriz inicial y luego el resto

		\asegura{
			$\{ \existe{p}{\TLista{\TLista{\TLista{\ent}}}}{ \left( p[0] = C_{0} \right) \implicaLuego \paraTodo{i,j,k}{\ent}{
						\left( \left( 0 < k < n \right) \wedge \left( 0 \leq i,j < |conexi \acute{o} n| \right) \right) \implicaLuego
						\left( conexi \acute{o} n[i][j] = multEntreMatrices(p[k-1], p[k]) \right) }  } \}$ \\
		}

		\}
	}

\end{proc}

\aux{multEntreMatrices}{mUno : \TLista{\TLista{\ent}}, mDos : \TLista{\TLista{\ent}}}{\ent}{ \paraTodo{i,j}{\ent}{\left( 0 \leq i,j < |mUno| \right) \yLuego \sum_{k=1}^{|mUno|} mUno[i][k] \times mDos[k][j]} }

%ambos (requiere y asegura) son falsos:

\subsection{caminoMinimo}

\begin{proc}{caminoMinimo}{\In origen : \ent, \In destino : \ent, \In distancias : \TLista{\TLista{\ent}}}{\TLista{\ent}}{ \{
		\requiere{$\{ \paraTodo{i,j}{\ent}{
					\left( 0 \leq i,j < |distancias| \right) \yLuego \left( \left( i = j \right) \longrightarrow \left( distancias[i][j] = 0 \right) \right) \wedge \left( distancias[i][j] = distancias[j][i] \right)
				}
				\}$}
		\asegura{$\{(hayCamino(distancias, origen, destino) \wedge \space ((\forall k: \TLista{\ent})(\exists p: \TLista{\ent})(esCamino(distancias, origen, destino, p)  \wedge \\ (sumaDistancias(distancias, origen, destino, p) \leq
				sumaDistancias(distancias, origen, destino, k)))
				\implica res = p
				)  \wedge \\ ((\neg hayCamino(distancias, origen, destino)
				\lor (origen = destino)) \implica res = \lvacia))
				\}$}
		\}
	}

	\pred{esCamino}{distancias : \TLista{\TLista{\ent}}, origen : \ent, destino : \ent, p : \TLista{\ent}}{$\{
			\paraTodo{k}{\ent}{ (0 \leq i, j, origen, destino  < |p|) \yLuego distancias[p[k]][p[k + 1]] } \}$
	}

	\aux{sumaDistancias}{distancias : \TLista{\TLista{\ent}}, origen : \ent, destino : \ent, s : \TLista{\ent}}{\ent}{$\sum_{j = origen}^{destino} \text{distancias}[j][j + 1]$}

\end{proc}

\section{Demostraciones de correctitud}
\vspace{0.1cm}
\subsection{Demostración de implementación}

\text{Definimos como precondicion y postcondicion de nuestro programa: } \\
$P \equiv \{( \exists i : \ent) (0 \leq i < |ciudades|) \yLuego ciudades[i].habitantes > 50.000 \wedge \paraTodo{j}{\ent}{ \left( 0\leq j < |ciudades| \implicaLuego ciudades[j].habitantes \geq 0 \right) \wedge \paraTodo{i,j}{\ent}{\left( 0\leq i,j < |ciudades| \implicaLuego ciudades[j].nombre \neq ciudades[i].nombre \right)} } \}$ \\
\vspace{0.1cm}
$Q \equiv \{\sum_{i=0}^{|ciudades|-1} ciudades[i].habitantes \}$ \\

% \begin{align}
%\begin{itemize}

\text{Para demostrar la correctitud del programa probamos que:} \\
\{P\}S1;S2;S3\{Q\}\; \text es válida, con: \\
\vspace{0.1cm}  \\
%arreglar despues
$\text{S1} \equiv res:= 0 \\
	\text{S2} \equiv i:=0 \\
	%esto esta feo
	\text{S3} \equiv $\\
w h i l e ( i \textless c i u d a d e s . l e n g t h ) do \\
r e s = r e s + c i u d a d e s [ i ] . h a b i t a n t e s \\
i = i + 1 \\
e n d w h i l e \\

\vspace{0.1cm}

\text{Se busca por monotonía demostrar que:} \\
\vspace {0.1cm} \\
$1. P \implica \textit{wp} \{S1;S2,P_{c}\} \\
	2. P_{c} \implica \textit{wp}\{S3,Q_{c}\}$ \\
\vspace {0.1cm} \\

$ \text{Siendo que } Q_{c}  \text{coincide con el final del código},  Q_{c} \equiv Q $\\
\text{Esto permite demostrar que P } \implica \textit{wp}\text{\{S1;S2;S3,Q\}}
\text{es verdadera por lo cual la Tripla de Hoare es válida}

\vspace{0.3cm}

$\text{Para demostrar } P_{c} \implica \textit{wp} \{{S3,Q_{c}\}} \text{se prueba la correctitud del ciclo mediante:}$ \\
\vspace*{1cm} \\
\textbf{Teorema del Invariante y Teorema de Terminación}

%\end{itemize}
% \end{align}

\textbf{Teorema del Invariante:}

Si existe un predicado I tal que:
% \begin{equation}
% \begin{aligned}

$1. \, P_{c} \implica I$

$2. \, \{I \wedge B\} S \{I\}$

$3. \, I \wedge \; \neg B \implica Q_{c}$
% \end{aligned}
% \end{equation}

Entonces el ciclo es parcialmente correcto respecto de la especificación

\vspace{0.3cm}
\begin{itemize}

	\item{Definimos}
	      %   \begin{align}

	      $P_{c}  \equiv \{ \text{res} = 0 \wedge i=0 \}           \\ \vspace{0.1cm} Q_{c}  \equiv
		      \{ \text{res} = \sum_{i=0}^{|ciudades|-1} ciudades[i].habitantes \} \\
		      \vspace{0.1cm} B  \equiv \{i < |ciudades| \}                       \\ \vspace{0.1cm} I  \equiv \{ 0
		      \leq i \leq |ciudades| \land \text{res} =
		      \sum\limits_{j=0}^{i-1}ciudades[j].habitantes \}$

	      %   \end{align}
\end{itemize}

\text{Según el teorema del invariante}

\begin{enumerate}
	\item $P_{c} \implica I$
\end{enumerate}
\begin{itemize}
	\item $0 \leq i \leq |ciudades| \equiv 0 \leq 0 \leq |ciudades|  $ %\checkmark
	\item $\text{res} = 0 \land i = 0 \implica \text{res} = \sum\limits_{j=0}^{i-1}ciudades[j].habitantes \equiv \sum\limits_{j=0}^{0-1}ciudades[j].habitantes = 0$ %\ $\checkmark$
\end{itemize}

$P_{c}$ \implica \text{I es válido}

\vspace{0.3cm}

\text{Ahora demuestro que:}\\  \vspace{0.4cm} 2. $  {\{I \land B\}} \text{S} {\{{I}\}
		\longleftrightarrow {I \land B} \implica \textit{wp(S, I)}}$

\begin{itemize}
	\item
	      \textit{wp(S, I)}
	      $\equiv \textit{wp}(\textbf{S1; S2}, I)  \\
		      \equiv \textit{wp}(\textbf{S1},\text{wp}(\textbf{S2,I}))  \\
		      \equiv \textit{wp}(\textbf{S1};\;\text{i:=i+1}\;,\textit{I}) \\
		      \equiv \textit{wp}(\text{res:= res + ciudades[j].habitantes; i:=i+1}, I) \\
		      \equiv wp\left( res:=res+ciudades[j].habitantes, wp\left( i:=i, (0 \leq i \leq |ciudades|) \land \text{res} = \sum\limits_{j=0}^{i-1}ciudades[j].habitantes \right) \right) \\
		      \equiv wp \left( res:=res + ciudades[j].habitantes, 0\leq i+1 \leq |ciudades| \wedge \sum\limits_{j=0}^{i+1-1}ciudades[j].habitantes \right) \\
		      \equiv wp \left( res:=res + ciudades[j].habitantes, -1\leq i \leq |ciudades|-1 \wedge \sum\limits_{j=0}^{i}ciudades[j].habitantes \right)\\
		      \equiv wp \left( res:=res + ciudades[j].habitantes, i < |ciudades| \wedge \sum\limits_{j=0}^{i-1}ciudades[j].habitantes + ciudades[i].habitantes \right)\\
		      \equiv i<|ciudades| \wedge res + ciudades[j].habitantes = \sum\limits_{j=0}^{i-1}ciudades[j].habitantes + ciudades[i].habitantes \\
		      \equiv i<|ciudades| \wedge res=\sum\limits_{j=0}^{i-1}ciudades[j].habitantes$
\end{itemize}
Ahora pruebo {I $\wedge$ B} \implica wp(S, I)
$I \wedge B \equiv 0 \leq i \leq |ciudades| \wedge res = \sum\limits_{j = 0}^{i-1}ciudades[j].habitantes
	\wedge i < |ciudades| \\ \equiv i < |ciudades| \wedge \sum\limits_{j = 0}^{i-1}ciudades[j].habitantes$

\begin{enumerate}
	\setcounter{enumi}{2}  % Esto hace que comience en 3
	\item ${\{I \land  \neg B\}} \implica\text{Qc}$

\end{enumerate}

\text{I $\wedge$ $\neg$ B} \\
$\equiv 0 \leq i \leq |ciudades| \land \text{res} = \sum\limits_{j=0}^{i-1}ciudades[j].habitantes \wedge \neg \left( i<|ciudades| \right) \\
	\equiv 0 \leq i \leq |ciudades| \land \text{res} = \sum\limits_{j=0}^{i-1}ciudades[j].habitantes \wedge i \geq|ciudades| \\
	\equiv \sum\limits_{j=0}^{|ciudades|-1}ciudades[j].habitantes$

Por el Teorema del Invariante podemos concluir que el ciclo es parcialmente
correcto respecto de la especificación

\textbf{Teorema de Terminación}

\begin{equation}
	\begin{align}
		4. & \{ I \wedge B \wedge fv= V_{o} \} \textbf{S} \{fv < V_{o} \} \\
		5. & ( I \wedge fv \leq 0) \implica \neg B
	\end{align}
\end{equation}

Elijo como función variante:

\begin{equation}
	{fv}= |ciudades| -i
\end{equation}

\begin{itemize}
	\item Verifico 4.
\end{itemize}

\begin{equation}
	\begin{align}
		$ & \{ I \wedge B \wedge fv= V_{o} \} \textbf{S} \{fv < V_{o} \} \longleftrightarrow \{ I \wedge B \wedge fv= V_{o} \} \implica \textit{wp} (S, fv < V_{o}) \\
		  & \textit{wp} (S, fv < V_{o}) \equiv \textit{wp} (res:= res + ciudades[i].habitantes; i:= i+1, |ciudades|-i < V_{o} )                                     \\
		  & \equiv \textit{wp}(res:= res + ciudades[i].habitantes, \textit{wp} (i:i+1; |ciudades|-i < V_{o} ))                                                      \\
		  & \equiv \textit{wp} (res:= res + ciudades[i].habitantes, |ciudades|-1-i<V_{o} )                                                                          \\
			\vspace{0.1cm}
		  & \text{Puedo ignorar la res ya que no es relevante en este caso}                                                                                         \\
		  & \equiv |ciudades| -1 -i < |ciudades| -i                                                                                                                 \\
		  & -1 < 0
			%\checkmark
		$
	\end{align}
\end{equation}

\vspace{0.3cm}
\begin{itemize}
	\item Verifico 5.
\end{itemize}

\begin{equation}
	\begin{align}
		 & ( I \wedge fv \leq 0) \implica \neg B                                                                                                                \\
		 & ( I \wedge fv \leq 0) \equiv  0 \leq i \leq |ciudades| \land \text{res} = \sum\limits_{j=0}^{i-1}ciudades[j].habitantes  \wedge |ciudades| -i \leq 0 \\
		 & \text{Ignoro res}                                                                                                                                    \\
		 & \equiv i \leq |ciudades| \wedge |ciudades| \leq i \ \longleftrightarrow |ciudades| = i \implica \neg (i < |ciudades|)                                \\
		 & \equiv|ciudades| = i \implica \neg (i < |ciudades|) \checkmark                                                                                       \\
		$$
	\end{align}
\end{equation}

Queda verificado el Teorema de Terminación por lo que podemos concluir que el
ciclo termina\\ Como ambos teoremas se cumplen, \textbf{el ciclo es correcto}.
\vspace{0.5cm}

\item Luego me falta ver que P \implica \textit{wp} \left( S1;S2,P_{c} \right) \\

wp \left(S1;S2,P_{c}\right) \equiv wp\left(res:=0; i:=0, P_{c} \right) \equiv
wp\left( res:=0,def(i) \yLuego P_{c}\space_{i}^{0} \right) \equiv wp(S_{2}, res
= 0 \wedge 0 = 0)) \\ \equiv wp(S_{2}, \True \yLuego res = 0))\equiv def(res)
\yLuego P_{c}\space_{res} ^{0} \equiv 0 = 0 \equiv \True \\

\\ Luego, como P \implica \space \True \space es tautología, siempre se cumple

\begin{equation}
\begin{align}
& P \implica \textit{wp} \{S1;S2;P_{c}\} \equiv \textit{wp} (S1,\textit{wp}(S2,P_{c})) \\
&\equiv \textit{wp}(S2, P_{c}) \equiv def (i) \yLuego P_{c} \equiv \textit{True} \yLuego P_{c} \\
&\equiv \textit{wp}(S1, P_{c})  \equiv def(res) \yLuego P_{c} \equiv \textit{True} \yLuego P_{c} \equiv P_{c} \\
& P \implica P_{c}

\item Queda demostrado que \textbf{la tripla \{P\}S1;S2;S3\{Q\}\; es válida}

\subsection{Demostracion res $>$ 50000}

\begin{multiline}
	\text{Se utiliza la misma definición del ejercicio anterior para $P,B,fv$. Se redefinen:  }\\
	I &\equiv \{ 0 \leq i \leq |ciudades| \land res = \sum \limits_{j=0}^{i-1}ciudades[j].habitantes \wedge ( \exists h : \ent) (0 \leq h < |ciudades|) \yLuego ciudades[h].habitantes > 50.000 \wedge \paraTodo{j}{\ent}{ \left( 0 \leq j < |ciudades| \right) \implicaLuego \left( ciudades[j].habitantes \geq 0 \right)}   \} \\
	Q &\equiv \{ \text{res} = \sum_{i=0}^{|ciudades|-1} ciudades[i].habitantes \wedge res>50000 \} \\
	\text{Se procede de manera similar al ejercicio anterior probando por monotonía la validez del programa y la tripla de Hoare.} \\
	\text{Resta chequear la validez del ciclo con el nuevo invariante y postcondicion. } \\
	\textbf{Teorema del invariante} \\
	P_{c} \longrightarrow I \\
	\text{Como se probo en el punto anterior, las implicancias para $0\leq i<|ciudades|$ y $\text{res} = \sum\limits_{j=0}^{i-1}ciudades[j].habitantes$ }\\
	\text{se analiza lo agregado al invariante. Como lo agregado esta presente en $P_{c}$ queda probado trivialmente que  $P_{c} \longrightarrow I$ }\\
	$\{I \land B\} S \{{I}\} \longleftrightarrow {I \land B} \implica wp(S, I) $\\
	\text{Lo agregado al invariante no afecta a la implicancia. Luego vale que: ${I \land B} \implica \textit{wp(S, I)}$}\\
	\vspace{0.2cm}\\
	\text{\{I \land  \neg B\} $\implica Q_{c}$}\\
	\text{$I \wedge \neg B \equiv  0\leq i\leq |ciudades| \wedge res = \sum\limits_{j=0}^{i-1} ciudades[j].habitantes \wedge ( \exists i : \ent) (0 \leq i < |ciudades|) \yLuego ciudades[i].habitantes>50.000 $} \\
	\text{$\wedge \paraTodo{j}{\ent}{ 0\leq j < |ciudades| \implicaLuego ciudades[j].habitantes \geq 0 } \wedge \neg \left( i<|ciudades| \right)$}\\
	\text{$\equiv 0\leq i\leq |ciudades| \wedge res = \sum\limits_{j=0}^{i-1} ciudades[j].habitantes \wedge ( \exists i : \ent) (0 \leq i < |ciudades|) \yLuego ciudades[i].habitantes>50.000 $} \\
	\text{$\wedge \paraTodo{j}{\ent}{ 0\leq j < |ciudades| \implicaLuego ciudades[j].habitantes \geq 0 } \wedge  i\geq|ciudades| $}\\
	\text{$\equiv i=|ciudades| \wedge res = \sum\limits_{j=0}^{i-1} ciudades[j].habitantes \wedge ( \exists i : \ent) (0 \leq i < |ciudades|) \yLuego ciudades[i].habitantes>50.000 $}\\
	\text{$\wedge \paraTodo{j}{\ent}{ 0\leq j < |ciudades| \implicaLuego ciudades[j].habitantes \geq 0 } $}\\
	\text{$\equiv res = \sum\limits_{j=0}^{|ciudades|-1} ciudades[j].habitantes \wedge ( \exists i : \ent) (0 \leq i < |ciudades|) \yLuego ciudades[i].habitantes>50.000 $}\\
	\text{$\wedge \paraTodo{j}{\ent}{ 0\leq j < |ciudades| \implicaLuego ciudades[j].habitantes \geq 0 } $}\\
	\vspace*{0.3cm}\\
	\text{Sea $k$ una posición en la sucesión $ciudades$ que cumple con la precondición de tener mas de 50000 habitantes}\\
	\text{$\sum\limits_{j=0}^{k-2} ciudades[j].habitantes$ Para los elementos a la izquierda de la k-esima posición}\\
	\text{n para el numero de habitantes de la k-esima posición }\\
	\text{$\sum\limits_{j=k+1}^{i-1} ciudades[j].habitantes$ Para los elementos a la derecha de la k-esima posición}\\
	\text{Luego }\\
	\text{$res = \sum\limits_{j=0}^{|ciudades|-1} ciudades[j].habitantes \wedge ( \exists i : \ent) (0 \leq i < |ciudades|) \yLuego ciudades[i].habitantes>50.000 $}\\
	\text{$\wedge \paraTodo{j}{\ent}{ 0\leq j < |ciudades| \implicaLuego ciudades[j].habitantes \geq 0 } \equiv res = \sum\limits_{j=0}^{|ciudades|-1} ciudades[j].habitantes \wedge $}\\
	\text{$res>50000$}\\
	\text{\{I \land  \neg B\} $\implica Q_{c}$}\\
	\text{Las implicancias del teorema de terminación no se ven afectadas por los cambios en el invariante y la postcondicion. Luego siguen siendo validas.}\\
	\text{Habiendo analizado todos los cambios y siguiendo los razonamientos del ejercicio anterior,}\\ \text{queda demostrado que el resultado devuelto es mayor a 50000}

\end{multiline}

\end{document}
